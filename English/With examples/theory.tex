\section{Theoretical Background}



This APA7 template for \LaTeX\ offers many features that will be demonstrated in this document. Just refer to the \LaTeX\ code to reproduce or modify these features by yourself.

As you have already seen, paragraphs, demarked by an empty line above and below in the source-script, are indented by 0.5in. 

The page is A4 size and the margins are set at 1in. 

\subsection{Formatting headings}
Headings of different levels can easily be used, numbering is automatic.
\subsection{Citation and Quotes in APA7-style}
For citations, the biblatex package is used with the biber backend and apa7 style. To cite a source, first enter the reference in BibTeX format into the references.bib file. For demonstration purposes, we will use two sources. Remember: A reference only appears in the text and in the bibliography, if it is referenced in the text.

\subsubsection{In Text citations in APA7 style}
To understand how this works, check the corresponding \LaTeX\ code. 

In this sentence, we add a reference in paranthesis after the sentence has ended \parencite{Tversky1974}. 

Here however, we note the contributions of \acite{Einstein1905} to science in the middle of the sentence. Note, that this command is defined only in this template and cannot be used without copying the definition. Tipp: You can click the year of the citation to jump to the bibliography entry. 

\subsubsection{Block quotes}
In APA7 style, any quote longer than 30 words should be formatted in an intended block

\begin{displayquote}
    This is such a block quote. Implementing this into you thesis is very easy thanks to \LaTeX. The indentation for these quotes can be changed in the main document's preamble.
\end{displayquote}
