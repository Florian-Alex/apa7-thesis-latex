% ==========================================
% Dokumentklasse: APA 7. Auflage für Studierende
% ==========================================
\documentclass[stu, floatsintext, 12pt, a4paper, donotrepeattitle]{apa7}

% ==========================================
% Sprache auf Deutsch, deutsche Anführungsstriche
% ==========================================
\usepackage[ngerman]{babel}
\usepackage[%
left = \glqq,%
right = \grqq,%
leftsub = \glq,%
rightsub = \grq%
]{dirtytalk}


% ==========================================
% Pakete für Grafiken und Tabellen
% ==========================================
\usepackage{graphicx} % Erforderlich für das Einfügen von Bildern
\usepackage{multirow} % Ermöglicht mehrzeilige Zellen in Tabellen
\usepackage{svg} % Ermöglicht das Einfügen von SVG-Bildern
\usepackage{tikz}
\usepackage{pgfplots}
\usetikzlibrary{intersections}
\usetikzlibrary{patterns}
\usepgfplotslibrary{fillbetween}
\pgfplotsset{compat=1.18}

% ==========================================
% Kurzfassung des Titels für die Kopfzeile
% ==========================================
\shorttitle{Ihr Kurztitel kommt hier hin}

% ==========================================
% Benutzerdefinierte Befehle für APA-konforme Statistiken
% ==========================================
\newcommand{\acite}[1]{{\citeauthor{#1} (\citeyear{#1})}} % Zitieren mit Autor und Jahr
\newcommand{\msd}[2]{($M$ = {#1}, $SD$ = {#2})} % Formatierung für Mittelwert und Standardabweichung
\newcommand{\chitest}[3]{$\chi^2$ (#2) = {#1}, $p$ ${#3}$} % Formatierung für den Chi-Quadrat-Test
\newcommand{\maineffect}[5]{$F$({#1}, {#2}) = {#3}, $p$ ${#4}$, $\eta^2_p$ = {#5}} % Formatierung für den Haupteffekt in der ANOVA
\newcommand{\levenes}[4]{$F$({#1}, {#2}) = {#3}, $p$ = {#4}} % Formatierung für den Levene-Test
\newcommand{\ttest}[4]{$t$({#1}) = {#2}, $p$ ${#3}$, $d$ = {#4}} % Formatierung für den t-Test
\newcommand{\shapiro}[2]{$W$ = {#1}, $p$ = {#2}} % Formatierung für den Shapiro-Wilk-Test

% ==========================================
% Formatierung von Abschnitten und Unterabschnitten
% ==========================================
\let\paragraph\oldparagraph
\let\subparagraph\oldsubparagraph
\usepackage{titlesec, blindtext, color} % Kontrolle über Abschnitts- und Unterabschnittsformatierung
\titleformat{\subsubsection}[runin]{\bfseries}{\hspace{.5in}\thesubsubsection \ }{0in}{}[]
\titleformat{\subsection}[block]{\bfseries}{\thesubsection \ }{0in}{}[]
\titleformat{\section}[block]{\hfil\bfseries}{\thesection \ }{0in}{}[]

% ==========================================
% Epigraphen und Zitate
% ==========================================
\usepackage{epigraph} % Für Zitate am Anfang von Kapiteln/Abschnitten
\setlength\epigraphwidth{\textwidth} % Lässt Epigraphen über die gesamte Textbreite laufen
\setlength\epigraphrule{.5pt} % Fügt eine kleine horizontale Linie unter dem Zitat hinzu
\usepackage{csquotes} % Bietet bessere Kontrolle über Anführungszeichen

% ==========================================
% Formatierung des Inhaltsverzeichnisses
% ==========================================
\setcounter{secnumdepth}{3} % Aktiviert Nummerierung bis zu Unterunterabschnitten
\usepackage[titles]{tocloft} % Anpassung des Inhaltsverzeichnisses
\renewcommand{\cftsecfont}{\normalsize} % Vereinheitlicht die Schriftgröße für Abschnitte im Inhaltsverzeichnis
\renewcommand{\cftsecpagefont}{\normalsize} % Vereinheitlicht die Schriftgröße für Abschnittsseitenzahlen im Inhaltsverzeichnis
\renewcommand{\cftsecleader}{\cftdotfill{.}} % Fügt Punkte zu Abschnittstiteln im Inhaltsverzeichnis hinzu
\renewcommand{\cftsubsecleader}{\cftdotfill{.}} % Fügt Punkte zu Unterabschnittstiteln im Inhaltsverzeichnis hinzu
\renewcommand{\cftsubsubsecleader}{\cftdotfill{.}} % Fügt Punkte zu Unterunterabschnittstiteln im Inhaltsverzeichnis hinzu
\cftsetindents{section}{0em}{1em}
\cftsetindents{subsection}{2em}{2em}
\cftsetindents{subsubsection}{3em}{3em}

% ==========================================
% Allgemeine Formatierung und Layout
% ==========================================
\usepackage{geometry} % Steuert Seitenränder
\geometry{margin=1in} % Setzt 1-Zoll-Ränder
\usepackage{makecell} % Ermöglicht dickere Zellränder in Tabellen
\usepackage[document]{ragged2e} % Rechtfertigt Absätze unter Beibehaltung von Zeilenumbrüchen
\setlength{\JustifyingParindent}{.5in} % Passt die Absatzeinrückung an
\usepackage[activate={true,nocompatibility},final,kerning=true,spacing=false,tracking=true]{microtype} % Verbessert die Textverteilung und Silbentrennung
\microtypecontext{spacing=nonfrench}
\hypersetup{hidelinks} % Versteckt die Farben von Hyperlinks im Inhaltsverzeichnis

% ==========================================
% Bibliographie und Referenzen
% ==========================================
\usepackage[style=apa, backend=biber]{biblatex} % Verwendet APA-Stil für Referenzen mit Biber
\addbibresource{references.bib} % Lädt die Bibliographiedatei

% ==========================================
% Schriftarten und Typografie
% ==========================================
%\usepackage{mathptmx} % Verwendet die Schriftart Times New Roman
\renewcommand{\arraystretch}{1.5} % Passt den Zeilenabstand in Tabellen an

% ==========================================
% Verschiedene Pakete
% ==========================================
\usepackage{lipsum} % Fügt Blindtext für Tests hinzu
\usepackage{setspace} % Ermöglicht manuelle Anpassung des Zeilenabstands
\usepackage{amsmath} % Verbessert die mathematische Notation

\begin{document}
\onehalfspacing
\justifying
\vspace*{4\baselineskip}
\thispagestyle{empty}
\begin{center}
   Titel Zeile 1\\Titel Zeile 2 \\
    \vspace*{5\baselineskip}
    \textbf{Bachelor Thesis} \\
    \vspace*{2\baselineskip}
    zur Erlangung des akademischen Grades \\
Bachelor of Science (B.Sc.) Psychologie \\
durch die Fakultät für Human- und Sozialwissenschaften der \\
Universitätsname \\
\vspace*{4\baselineskip}
\end{center}
vorgelegt von \\
\noindent Vor- und Nachname\\
\noindent \hspace{.5in} Straße \\
\noindent\hspace{.5in} PLZ, Ort \\
\noindent Matrikelnummer: 123456 \\
\noindent E-Mail: name@uni.de \\
\noindent Datum: 26. April 2025 \\
\begin{tabbing}
     \hspace*{1cm}\=\hspace*{1.5in}\=\hspace*{4cm}\=\hspace*{2.7cm}\= \kill
     Prüfer/in und Betreuer: \> \> Prof. Dr. Max Mustermann \\
     Prüfer/in: \> \>  Dr. Maria Musterfrau
\end{tabbing}
\section*{Abstract}
\noindent Your abstract goes here, around 250 words.
\newpage

\tableofcontents

\newpage
\section{Theoretischer Hintergrund}
\say{Test}.
\section{Methodology}
\subsection{Tables}
Tables are vital for any thesis. While \LaTeX\ does not offer the same ease in mocking up tables quickly, the tables turn out much neater and less often destroy your layout than in Word or similar programs. Writing the tables entirely by hand is not recommended, it is much faster and more seamless to use online \LaTeX\ table generation sites. Then just copy the code here and customize.  

Note, that tables in APA7 should use horizontal lines sparingly and avoid vertical lines entirely. The APA7 documentclass numbers the tables automatically and the captions and tablenotes are italized as required. This is what a table in APA7 style can look like:

\begin{table}[h]
    \caption{Descriptive characteristics of experimental groups on Big5 factors}
\label{big5}
    \begin{tabular}{lcc}
    \hline
     \textbf{Factor}& \multicolumn{2}{c}{\textit{\textbf{M {[}SD{]}}}} \\ \hline
     & Group A & Group B \\
    Agreeableness & 7.13 {[}2.11{]} & 10.95 {[}1.74{]} \\
    Extraversion & 3.54 {[}0.54{]} & 2.22 {[}4.87{]} \\
    Openness & 4.33 {[}6.32{]} & 1.64 {[}3.22{]} \\
    Neuroticism & 2.59 {[}3.99{]} & 5.66 {[}1.16{]} \\
    Concioussness & 3.88 {[}1.04{]} & 3.32 {[}4.38{]} \\ \hline
    \end{tabular}
    \tablenote{Data are made up.}
    \end{table}
\section{Results}
\subsection{Figures}
\subsubsection{Mathematical Figures}
Because the primary usecase for \LaTeX\ is the usage in natural sciences and STEM subjects, there are features to display complex mathematical illustrations. 

\begin{figure}
    \caption{Visualization of a normal distribution with the 95\% confidence interval shaded in blue and rejection regions in red.}
    \centering
    \begin{tikzpicture}
        \begin{axis}[
            axis lines=middle,
            enlargelimits=upper,
            domain=-3:3,
            samples=100,
            xtick={-3,-2,-1,0,1,2,3},
            xticklabels={$-3$,$-2$,$-1$,$0$,$1$,$2$,$3$},
            ytick=\empty,
            xlabel={$z$},
            ylabel={Density},
            every axis x label/.style={at={(current axis.right of origin)},anchor=north west},
            every axis y label/.style={at={(current axis.above origin)},anchor=south east}
        ]
        
        % Normal distribution curve
        \addplot[name path=N, domain=-3:3, samples=100, thick] {exp(-x^2/2)/sqrt(2*pi)};
        
        % Confidence interval (shaded region between -1.96 and 1.96)
        \path[name path=lower] (-1.96,0) -- (1.96,0);
        \addplot[blue!30] fill between[of=N and lower, soft clip={domain=-1.96:1.96}];
        
        % Left and right rejection regions
        \path[name path=left] (-3,0) -- (-1.96,0);
        \path[name path=right] (1.96,0) -- (3,0);
        \addplot[red!50, pattern=north east lines, pattern color=red] fill between[of=N and left, soft clip={domain=-3:-1.96}];
        \addplot[red!50, pattern=north east lines, pattern color=red] fill between[of=N and right, soft clip={domain=1.96:3}];
        
        % Vertical dashed lines for z = -1.96 and z = 1.96
        \addplot[dashed] coordinates {(-1.96,0) (-1.96,0.2)};
        \addplot[dashed] coordinates {(1.96,0) (1.96,0.2)};
        
        % Labels
        \node[anchor=north] at (axis cs:-1.96,0) {$-1.96$};
        \node[anchor=north] at (axis cs:1.96,0) {$1.96$};
        \node at (axis cs:0,0.05) {95\% CI};
        
        \end{axis}
    \end{tikzpicture}
    \label{fig:confidence_interval}
\end{figure}
\newpage
\subsubsection{External Figures}
External image files like .png or .jpg images can easily be added and automatically numbered as well. Note that the image file should be in the project folder to refer to it.

\begin{figure}
    %\centering
     \caption{An example figure}
     \includegraphics[width=\linewidth]{Tables & Figures/DCTplot.jpg}
     
     %\label{fig:enter-label}
 \end{figure}
\include{discussion}
\newpage

\newpage
\section{Referenzen}
\printbibliography[heading=none]

\appendix
\newpage
\addcontentsline{toc}{section}{Anhang}
\section{Additional material}
If there is one section in the appendix, the appendix is not enumerated. However, if you add another section to the appendix, they are then labeled alphabetically.

\newpage
\newpage
\begin{center}
    \noindent \large\textbf{Eigenständigkeitserklärung}
\end{center}

\vspace{1cm}
\noindent Ich, YOUR NAME, erkläre hiermit, dass diese Abschlussarbeit mit dem Titel „[Thesis Title]“ meine eigene Arbeit ist. Ich habe keine anderen Quellen, Hilfsmittel oder Unterstützung verwendet als die ausdrücklich im Text angegebenen.\vspace{.3cm}

\noindent Alle von mir zitierten, paraphrasierten oder anderweitig verwendeten Quellen sind ordnungsgemäß angegeben. Diese Arbeit entspricht den Richtlinien zur wissenschaftlichen Integrität der [University Name].

\vspace{35mm}
\begin{tabular}{@{}p{2in}p{3.5in}@{}}
Stadt, 26. April 2025 & \hrulefill\\
& \centering Ihr Vor- und Nachname \\
\end{tabular} 


\end{document}